%%%%%%%%%%%%%%%%%%%%%%%%%%%%%%%%%%%%%%%%%
% Twenty Seconds Resume/CV
% LaTeX Template
% Version 1.1 (8/1/17)
%
% This template has been downloaded from:
% http://www.LaTeXTemplates.com
%
% Original author:
% Carmine Spagnuolo (cspagnuolo@unisa.it) with major modifications by 
% Vel (vel@LaTeXTemplates.com)
%
% License:
% The MIT License (see included LICENSE file)
%
%%%%%%%%%%%%%%%%%%%%%%%%%%%%%%%%%%%%%%%%%

%----------------------------------------------------------------------------------------
%	PACKAGES AND OTHER DOCUMENT CONFIGURATIONS
%----------------------------------------------------------------------------------------

\documentclass[letterpaper]{twentysecondcv} % a4paper for A4

%----------------------------------------------------------------------------------------
%	 PERSONAL INFORMATION
%----------------------------------------------------------------------------------------

% If you don't need one or more of the below, just remove the content leaving the command, e.g. \cvnumberphone{}

\profilepic{FotoQuadrata.jpg} % Profile picture

\cvname{Roberto Allocco} % Your name
\cvjobtitle{Consulente Aziendale} % Job title/career

\cvdate{} % Date of birth
\cvaddress{} % Short address/location, use \newline if more than 1 line is required
\cvnumberphone{+39 3397250493} % Phone number
\cvsite{} % Personal website

% Avoid scrapers
\def\maila{allocco.}
\def\mailb{roberto}
\def\mailc{@gma}
\def\maild{il.com}
\cvmail{\maila\mailb\mailc\maild} % Email address

%----------------------------------------------------------------------------------------

\begin{document}


%----------------------------------------------------------------------------------------

\makeprofile % Print the sidebar

%----------------------------------------------------------------------------------------
%	 INTERESTS
%----------------------------------------------------------------------------------------

\section{Abilità}
\\Sono un consulente tecnico e sviluppatore.
Ho \textbf{portato a termine più di 40 progetti di digitalizzazione} per aziende in settori estremamente variegati (\underline{Automotive,}  \\\underline{Cooperative, Produzione Meccanica,} \underline{Studi Commercialisti, Studi di Comunicazione,} \\\underline{Società Edilizie ecc}), in altrettanti ambiti (\textit{Gestione del personale, Gestione dei Cicli Attivo e Passivo, Controllo dei Progetti,}\textit{ Controllo della Produzione ecc}).\\
Ho sempre usato piattaforme fortemente dipendenti da \textbf{SQL}, integrandole con plugin e applicativi sviluppati da me (\underline{Python, Powershell,} \underline{AngularJS}), affinché rispondessero alle mie esigenze.
https://it.overleaf.com/project/65550a7407402a13e88b12d9
%----------------------------------------------------------------------------------------
%	 EXPERIENCE
%----------------------------------------------------------------------------------------

\section{Esperienze}

\begin{twenty} % Environment for a list with descriptions
	\twentyitem{2021-2023}{Consulente Senior e Sviluppatore}{Pragmos}{Sviluppo di progetti di digitalizzazione su piattaforme web based, con \textbf{integrazioni sviluppate in AngularJS.}}
	\twentyitem{2017-2020}{Consulente Senior}{}{Supervisione e gestione di piccoli team di sviluppo (< 5 persone) per progetti di digitalizzazione. \textbf{Sviluppo di integrazioni in python.}}
	\twentyitem{2012-2016}{Consulente Tecnico}{}{Gestione integrale di progetti di digitalizzazione su applicativi Windows Form, \textbf{integrazioni sviluppate tramite PowerShell}. Analisi ed estrazione di algoritmi da software per la previsione della domanda, scritto in \textbf{Object Pascal}. Applicazione dei suddetti ad ambienti di produzione.}
\end{twenty}


%----------------------------------------------------------------------------------------
%	 EDUCATION
%----------------------------------------------------------------------------------------
\section{Educazione}

\begin{twenty} % Environment for a list with descriptions
	\twentyitemsmall{2006-2012}{Laurea In Ingegneria Gestionale - Produzione Industriale}{Politecnico di Torino}
	\twentyitemsmall{2000-2005}{Perito Meccanico}{I.T.I.S Bra}
	%\twentyitem{<dates>}{<title>}{<location>}{<description>}
\end{twenty}

%----------------------------------------------------------------------------------------
%	 OTHER INFORMATION
%----------------------------------------------------------------------------------------

\section{Projects and Publications}

\begin{twenty} % Environment for a list with descriptions
	\twentyitem{2022}{None}{None}{
		nothing to see here yet.
	 }
\end{twenty}
%
%----------------------------------------------------------------------------------------
%	 SECOND PAGE EXAMPLE
%----------------------------------------------------------------------------------------
%
%\newpage % Start a new page
%
%\makeprofile % Print the sidebar
%
%\section{Other information}
%
%\subsection{Review}
%
%Alice approaches Wonderland as an anthropologist, but maintains a strong sense of noblesse oblige that comes with her class status. She has confidence in her social position, education, and the Victorian virtue of good manners. Alice has a feeling of entitlement, particularly when comparing herself to Mabel, whom she declares has a ``poky little house," and no toys. Additionally, she flaunts her limited information base with anyone who will listen and becomes increasingly obsessed with the importance of good manners as she deals with the rude creatures of Wonderland. Alice maintains a superior attitude and behaves with solicitous indulgence toward those she believes are less privileged.
%
%\section{Other information}
%
%\subsection{Review}
%
%Alice approaches Wonderland as an anthropologist, but maintains a strong sense of noblesse oblige that comes with her class status. She has confidence in her social position, education, and the Victorian virtue of good manners. Alice has a feeling of entitlement, particularly when comparing herself to Mabel, whom she declares has a ``poky little house," and no toys. Additionally, she flaunts her limited information base with anyone who will listen and becomes increasingly obsessed with the importance of good manners as she deals with the rude creatures of Wonderland. Alice maintains a superior attitude and behaves with solicitous indulgence toward those she believes are less privileged.
%
%----------------------------------------------------------------------------------------
\end{document} 